\documentclass{article}
\usepackage{polski}
\usepackage[utf8]{inputenc}
\usepackage[T1]{fontenc}
\usepackage{graphicx}
\usepackage{indentfirst}  
\usepackage[polish]{babel}
\usepackage[left=2.5cm,right=2.5cm,top=2cm,bottom=2cm]{geometry}




\begin{document} %---------------------------------------------------------------------------------------------------------------------% 

\centerline{\textbf{\Large Seminarium magisterskie}}
\vspace{10mm}
{\Large Wykorzystanie cyfrowej analizy obrazu w metodykach pracy z dziećmi w wieku przedszkolnym}

\vspace{10mm}
\centerline{\large Jakub Lebiedziński} 
\vspace{1mm}
\centerline{\normalsize {269096}} 
\vspace{5mm} 

\linespread{1.3} % interlinia dla dalszej zawartości dokumentu = 1.5
\large % wielkość czcionki dla dalszej zawartości dokumentu

\section*{\textbf{Wstęp}}
Sed ut perspiciatis unde omnis iste natus error sit voluptatem accusantium doloremque laudantium, totam rem aperiam, eaque ipsa quae ab illo inventore veritatis et quasi architecto beatae vitae dicta sunt explicabo. Nemo enim ipsam voluptatem quia voluptas sit aspernatur aut odit aut fugit, sed quia consequuntur magni dolores eos qui ratione voluptatem sequi nesciunt. Neque porro quisquam est, qui dolorem ipsum quia dolor sit amet, consectetur, adipisci velit, sed quia non numquam eius modi tempora incidunt ut labore et dolore magnam aliquam quaerat voluptatem. Ut enim ad minima veniam, quis nostrum exercitationem ullam corporis suscipit laboriosam, nisi ut aliquid ex ea commodi consequatur? Quis autem vel eum iure reprehenderit qui in ea voluptate velit esse quam nihil molestiae consequatur, vel illum qui dolorem eum fugiat quo voluptas nulla pariatur? \cite{ref1}
\par




\newpage %każdy punkt główny od nowej linii
\renewcommand\refname{\section*{Bibliografia}}
\begin{thebibliography}{11}\linespread{1}\normalsize{
%{11} - max liczba źródeł bibliograficznych
	\bibitem{ref1} 
	Science Daily, Big Data, for better or worse: 90\% of world’s data generated over last two years, 2013
	\bibitem{ref2}	
	A. Geron, Uczenie maszynowe z użyciem Scikit-Learn i TensorFlow, Helion, 2018
	\bibitem{ref3}	
	F.Chollet, Deep Learning. Praca z językiem Python i biblioteką Keras, Helion 2019
	\bibitem{ref4}	
	Y. Karayaneva, D. Hintea, Object Recognition in Python and MNIST Dataset Modification and Recognition with Five Machine Learning Classifiers, Journal of Image and Graphics, Vol. 6, No. 1, June 2018
	\bibitem{ref5}
	R.  Kohavi, F.  Provost, Glossary  of  terms, Applications  of Machine Learning and the Knowledge Discovery Process, vol. 30, pp. 271-274, 1998
	\bibitem{ref6}
	J.Grus, Data science od podstaw, Helion, 2018
	\bibitem{ref7}
	R. B. Arif, M. A. B. Siddique, M. M. R. Khan, M. R. Oishe, Study  and  Observation  of  the  Variations  of  Accuracies  for Handwritten  Digits  Recognition  with  Various  Hidden  Layers and Epochs using Convolutional Neural Network, in 2018 4th International  Conference  on  Electrical  Engineering  and Information  and  Communication  Technology  (iCEEiCT),  2018, pp. 112-117 EEE
	\bibitem{ref8}
	E. Kussul, T. Baidyk, Improved method of handwritten digit recognition  tested  on  MNIST  database,  Image  and  Vision Computing, vol. 22, no. 12, pp. 971-981, 2004
	\bibitem{ref9}
	J. Jin, K. Fu, and C. Zhang, Traffic sign recognition with hinge loss trained convolutional neural networks, IEEE Transactions on Intelligent Transportation Systems, vol. 15, no. 5, pp. 1991-2000, 2014
	\bibitem{ref10}
	A.Boschetti, L. Massaron, Python. Podstawy nauki o danych, Helion 2016
	\bibitem{ref11}
	Y.  LeCun,  K.  Kavukcuoglu, C.  Farabet, Convolutional  networks  and  applications  in  vision, in Proceedings of 2010 IEEE International Symposium on Circuits and Systems, 2010, pp. 253–256
}
\end{thebibliography}

\end{document}
