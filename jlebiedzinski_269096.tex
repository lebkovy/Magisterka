\documentclass{article}
\usepackage{polski}
\usepackage[utf8]{inputenc}
\usepackage[T1]{fontenc}
\usepackage{graphicx}
\usepackage{indentfirst}  
\usepackage[polish]{babel}
\usepackage[left=2.5cm,right=2.5cm,top=2cm,bottom=2cm]{geometry}




\begin{document} %---------------------------------------------------------------------------------------------------------------------% 

\centerline{\textbf{\Large Seminarium magisterskie}}
\vspace{10mm}
{\Large Wykorzystanie cyfrowej analizy obrazu w metodykach pracy z dziećmi w wieku przedszkolnym}

\vspace{10mm}
\centerline{\large Jakub Lebiedziński} 
\vspace{1mm}
\centerline{\normalsize {269096}} 
\vspace{5mm} 

\linespread{1.3} % interlinia dla dalszej zawartości dokumentu = 1.5
\large % wielkość czcionki dla dalszej zawartości dokumentu

\section*{\textbf{Wstęp}}
Przenieśmy się do roku 1945, właśnie dokonuje się przełom w dziedzinie cyfryzacji i komputeryzacji - pierwszy komputer ujrzał światło dzienne. Enigmatyczna nazwa ENIAC skłania nas do myślenia nad genezą tego słowa. W swojej istocie, nazwa ta, jest tak naprawdę banalna i pochodzi z języka angielskiego od ‘electronic numerical integrator and computer’ czyli ‘elektroniczny integrator numeryczny i komputer’. Robi się ciekawiej. Warto w tym miejscu nakreślić ogólne pojęcie o  zautomatyzowanych procesach w 1945 roku. Mam na myśli to, jak ludzie postrzegali wówczas maszyny wykonujące skomplikowane obliczenia matematyczne w znacznie krótszym czasie niż oni sami. Ludzkość w późniejszych latach ‘40 ubiegłego wieku była sceptycznie nastawiona do robotów i maszyn. Głównie dlatego, że ich nie rozumieli - taka natura ludzka, że boimy się tego czego nie jesteśmy wyjaśnić w logiczny sposób.
Dzisiaj tj. 80 lat później nie wyobrażamy sobie życia bez współczesnych rozwiązań techniki. Cóż, przynajmniej uważamy to za coś nierealnego, nieosiągalnego. Jak zauważa Monika Frania 'Człowiek żyjący na przełomie XX i XXI w. to świadek eksplozji przemian instrumentarium medialnego'. Ciężko się nie zgodzić. Nasi dziadkowie pamiętają czasy, w których technologia była im zupełnie obca. Ba, sami doświadczyliśmy okresu, w którym ‘Internet’ było słowem nieznanym, niezrozumiałym. Czym jest tak naprawdę ta cała ‘technologia’, o którą się tu rozchodzi? Według Polskiego Słownika PWN słowo ‘technologia’ opiera się na metodach powiązanych z procesem przetwórczym lub produkcji. Technologie informacyjne jako termin są z nami stosunkowo od niedawna. Wraz ze wzrostem znaczeń maszyn i komputerów, przez gromadzenie i agregację danych, aż do ‘targetowania’ użytkowników wspomnianego już Internetu, na znaczeniu zyskały również inne dziedziny nauki, ale i nie tylko.
Edukacja jest z nami od zarania dziejów. Już w starożytnej Grecji powstawały pierwsze ‘szkoły’, gdzie słuchacze mogli śledzić wykłady najznamienitszych filozofów ówczesnego świata. Platon, Arystoteles czy Sokrates byli uważani za mistrzów retoryki, etyki czy filozofii. Warto wspomnieć, że nie każdy obywatel antycznego państwa miał obowiązek pobierania nauki. Edukacja była bowiem przywilejem, na który stać było tylko najbogatszych. Dzisiaj jest to nie do pomyślenia. Z czasem jednak edukacja stawała się coraz bardziej dostępna dla niższych warstw społecznych. Dawni nauczyciele wspomagali się w edukowaniu swoich słuchaczy wskaźnikami, elementami natury, mapami, malunkami i tak dalej. Podobnie jest i teraz. Różnicą jest stopień zaawansowania ‘pomagaczy’ i procent ich wykorzystania w szkolnictwie.





\newpage %każdy punkt główny od nowej linii
\renewcommand\refname{\section*{Bibliografia}}
\begin{thebibliography}{11}\linespread{1}\normalsize{
%{11} - max liczba źródeł bibliograficznych
	\bibitem{ref1} 
	Nowe media, technologie i trendy w edukacji - Frania M.
	\bibitem{ref2}	
	Nowe technologie informacyjne w edukacji - Adamkiewicz J.
	\bibitem{ref3}	
	Multimedia w kształceniu - Bednarek J.
	\bibitem{ref4}	
	Komputer jako środek dydaktyczny w edukacji wczesnoszkolnej - Hassa A.
	\bibitem{ref5}
	Media w edukacji - Gajda J.
	\bibitem{ref6}
	Słownik terminów i pojęć badań jakościowych nad edukacją - Jagieła J.
}
\end{thebibliography}

\end{document}
